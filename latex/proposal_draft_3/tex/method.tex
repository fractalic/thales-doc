\subsection{\label{sec:method:theory}Theory}
There are numerous potential methods of detecting floating and undersea obstacles. Though these have rarely been applied to autonomous detection of objects, the functionality and expected usability of each system is discussed below. Also, general detection principles applicable to many detection schemes are discussed.

\subsubsection{\label{sec:method:theory:obstacles}Obstacles}
To determine the most important elements of our design, it is important to understand the magnitude of the risk posed by various types of floating hazards. Some of these hazards are detailed in Table \ref{tab:obstacles}, where it becomes apparent that AIS, detecting large boats, is the most critical piece of our obstacle detection scheme. The derivation of the frequency estimates is detailed in Appendix \ref{app:obstacle-frequency}.

\begin{table}
\caption{\label{tab:obstacles}Obstacles and risk}
\begin{tabular}{c|c|c|c}

Obstacle & Location & Frequency & Damage\\[0.6cm]
\hline

Shipping vessels & Entire route &200 km$^2$ & Severe \\
Icebergs & Entire route & 10000 km$^2$ & Severe \\
Logs & Entire route, drawn to gyres & Low & Moderate \\

\end{tabular}
\end{table}

\subsubsection{\label{sec:method:theory:passive-infrared-detectors}Passive infrared detectors}
Passive IR detectors examine their surroundings according to temperature. Given no IR emission from a beacon, the intensity of radiation emitted by an object depends on its temperature and emissivity.

Low-floating objects will typically have a temperature near the temperature of the water, however they are visible if their emissivity is different from that of water. Liquid water has an emissivity of $.90$ to $.95$; ice has an emissivity of $.95$ to $1$, making it appear white in an IR image against a water background; and wood has an emissivity of $.75$ to $.95$, making it appear black against water. \cite{optotherm-emisstable}


\subsection{\label{sec:method:proposed-design}Proposed Design}
The design that will be most likely to see all obstacles at appropriate distances will rely on AIS for large boats, IR for small boats,and  a small radar or radio reciever for nets. AIS is an easily implemented solution that rely solely on external data (no on-board sensors) and coding for positioning. The AIS module has already been purchased by the Sailbot team, and so its price and power budget are already taken care of- the module simply needs to be implemented. The radio receiver needs to identify the direction of a radio source to find a net marker, and avoiding that area should keep the boat away from any nets. The IR detection will be the most difficult of the individual systems to test and implement, and the true challenge will be integrating all three systems to appropriately manage data from each source.

Most of the teams effort will be directed towards implementing the IR camera. From our preliminary testing it seems that the IR camera will be the primary detection method, as it is able to clearly see boats at a large range, small floating obstacles at medium range, and the nighttime detection is excellent. Further, while AIS is useful around the coast, for nearly the full duration of the Sailbot's trip from the start to the finish line any boat encounters will be exceedingly rare. Images from an online AIS tracking source, shown below, shows this. Further, tracking all journeys across the north Atlantic during the third quarter of 2013 (the same time frame as the Sailbot will launch in 2015) shows the most common shipping lanes and the rarity of ships outside of these major lanes \cite{marine-traffic}.

\begin{figure}
\includegraphics[width=120mm,natwidth=540,natheight=459]{"./image/AIS_emitters_north_atlantic"}
\caption[All AIS emitters in the North Atlantic on Wednesday, November 26 2014.]{\label{fig:ais-emitters}Blocks show how many boats were uploading position data via AIS transmitters in a given area. The north Atlantic shows next to no activity in the prosposed path, except near the shores. There were over 59,000 worldwide emitters on this day.}
\end{figure}

\begin{figure}
\includegraphics[width=120mm,natwidth=540,natheight=459]{"./image/AIS-tracking-Q3-2013"}
\caption[AIS- captured tracking data for Q3 2013]{\label{fig:ais-emitters-amortized}All routing data sent via AIS during the third quarter of 2013.}
\end{figure}


The proposed design for the obstacle detection with the Sailbot team will incorporate, in order of most important to least:
\begin{itemize}
\item IR beacon and detector, providing a 2D image for small boat and obstacle detection
\item Radar module OR a simple radio receiver, to find radio transponders placed on net markers
\item AIS, an internationally managed system for large boats to communicate positions, headings and speeds
\end{itemize}

\subsubsection{\label{sec:method:proposed-design:ais}The AIS system}
The most straightforward method to detect the position and velocity of nearby large boats is with an AIS receiver. Typically, boats of all types (tankers, cargo ships, and passenger vessels) longer than 20m are required to carry an AIS emitter \cite{us-ais-requirements}. We require only a receiver, such as the EasyAIS A025 sold by WeatherDock (shown in figure \ref{fig:easyais-a025}), and a VHF antenna such as the SS-2 sold by GAM Electronics \cite{gam-electronics__ss-2}, connected to our Raspberry Pi, running a copy of the AIS signal decoder from the GPSD project.

The signals emitted by any AIS-compliant device are sent as serial data in the NMEA (National Marine Electronics Association) 0183 or 2000 format \cite{nmea__standards}. This device uses NMEA 0183, which can be easily decoded in Python using the premade library written by the GPSD project \cite{gpsd__overview}.

The receiver weighs 400g, measures 140mm by 140mm by 35mm, and consumes 720 mW of power during operation. The antenna weighs 114g, and is 122 cm tall. The sailbot can easily support this size and weight, though the high power draw of the receiver means that this device can only be operated sporadically. With an estimated range of 20km, a readout every 20 minutes would be sufficient to identify boats with a relative velocity of 60km/h -- the estimated speed of most cargo vessels of 40 km/h plus the maximum velocity of the sailbot equal to 20 km/h.

\begin{figure}
\includegraphics[width=80mm,natwidth=540,natheight=459]{"./image/ais-a025"}
\caption[AIS A025 by WeatherDock.]{\label{fig:easyais-a025}AIS A025, an inexpensive AIS receiver manufactured by WeatherDock. The device emits serial signals based on a variant of the NMEA serial communication standard.}
\end{figure}

The A025 is a relatively inexpensive device, retailing for about \$500. 


\subsection{\label{sec:method:proposed-analysis}Proposed Analysis}
It is important to verify the correct functioning both of the hardware sensors and computation platform, and of the software analysis system. Beyond functionality, further examination will determine the limits of the capabilities of the device.

The Sailbot team has previously competed with small sailboats. They have a small sailboat (approximately 3 feet long) that's completely built and thoroughly tested. We have full access to this boat and freedom to modify it to mount various detection systems in order to perform scale testing on it. This small boat will serve as a robust testing mechanism to try different orientations and combinations of systems.

\subsubsection{\label{sec:method:proposed-analysis:software}Software}
Software can be examined at a basic level by following the principles of test driven development, and at a higher level by providing simulated or offline signals to the program. The integration of the various systems used in the final solution can be tested via output to the mini Sailbot navigation system and navigation in a pool.

\subsubsection{\label{sec:method:proposed-analysis:hardware}Hardware}
The Sailbot has ample room in the hull of the boat for fairly large systems. these systems must be robust, immune to shaking and waterproof. Marine-grade cables that are standard use by the Sailbot team will serve as the primary connections from the sensors to the black-box detection electronics.


\subsection{\label{sec:method:evaluation-of-alternative-designs}Evaluation of Alternative Designs}

\subsection{\label{sec:method:proposed-verification-procedure}Proposed Verification Procedure}

\subsubsection{\label{sec:method:proposed-verification-procedure:software}Software}
For more information on software testing methodology, refer to Appendix \ref{app:software-testing-methodology}.

\subsubsection{\label{sec:method:proposed-verification-procedure:hardware}Hardware}
Hardware systems should be tested in isolation to verify that they produce the correct output for various inputs. Then the systems can be couple together, and software control added. The hardware is then tested as part of the software "user tests".

The hardware requiring testing will be several emitters and detectors, possibly of various types. The total power draw will be checked, along with physical and electronic output.
