\subsection*{Appendix A:\\Calculations for\\electromagnetic radiation emitters} \label{app:AppendixA}
% the \\ insures the section title is centered below the phrase: AppendixA
\subsubsection*{\label{sec:app:detection-principles}Minimum detection distance}
Constraints on the viewing area appear with any method of detection, whether radar, lidar, on infrared. For directional emitters, the minimum detection range is given by trigonometry and depends only on the vertical beam width and the mounting angle of the device. $d = h/\text{tan}(\eta)$, where $\eta=\theta/2+\phi$ given by $\theta$, the beam vertical beam spread, and $\phi$ the emitter mount angle.
\begin{figure}
\includegraphics[width=70mm,natwidth=505,natheight=394]{"./image/directional-emitters"}
\caption{\label{fig:emitter-angle}Minimum detection distance for a directional emitter.}
\end{figure}
