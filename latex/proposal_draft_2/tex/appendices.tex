\subsection*{Appendix A:\\Detection principles for\\electromagnetic radiation detectors} \label{app:AppendixA}
% the \\ insures the section title is centered below the phrase: AppendixA
\subsubsection*{\label{sec:app:min-distance}Minimum detection distance}
Constraints on the viewing area appear with any method of detection, whether radar, lidar, on infrared. For directional emitters, the minimum detection range is given by trigonometry and depends only on the vertical beam width and the mounting angle of the device. $d = h/\text{tan}(\eta)$, where $\eta=\theta/2+\phi$ given by $\theta$, the vertical beam spread, and $\phi$, the emitter mount angle.
\begin{figure}
\includegraphics[width=70mm,natwidth=505,natheight=394]{"./image/directional-emitters"}
\caption[Minimum distance to detected obstacle.]{\label{fig:emitter-angle}Minimum detection distance for a directional emitter.}
\end{figure}

\subsubsection*{\label{sec:app:infrared-image}Passive infrared detector notes}
Passive IR detectors examine their surroundings according to temperature. Given no IR emission from a beacon, the intensity of radiation emitted by an object depends on its temperature and emissivity.

Low-floating objects will typically have a temperature near the temperature of the water, however they are made visible if their emissivity is different from that of water. Liquid water has an emissivity of $.90$ to $.95$; ice has an emissivity of $.95$ to $1$, making it appear white in an IR image against a water bakcground; and wood has an emissivity of $.75$ to $.95$, making it appear black against water. \cite{optotherm-emisstable}
