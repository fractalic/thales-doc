\subsection{\label{sec:method:theory}Theory}
There are numerous potential methods of detecting floating and undersea obstacles. Though these have rarely been applied to autonomous detection of objects, the functionality and expected usability of each system is discussed below. Also, general detection principles applicable to many detection schemes are discussed.


\subsection{\label{sec:method:proposed-design}Proposed Design}
It seems likely that a radar/lidar combination will be our primary detection method.  This will be reassessed after meeting with our sponsor.


\begin{itemize}
\item LIDAR system, where laser rangefinders are used to create a 3d map of the surrounding space
\item IR beacon and detector, providing a 2D image
\item Ultrasound. 3D image
\item Other things?
\end{itemize}



\subsection{\label{sec:method:proposed-analysis}Proposed Analysis}
It is important to verify the correct functioning both of the hardware sensors and computation platform, and of the software analysis system. Beyond functionality, further examination will determine the limits of the capabilities of the device.
\subsubsection{\label{sec:method:proposed-analysis:software}Software}
Software can be examined at a basic level by following the principles of test driven development, and at a higher level by providing simulated or offline signals to the program.

\subsubsection{\label{sec:method:proposed-analysis:hardware}Hardware}
After meeting with our sponsor, we will develop a test plan to verify that our hardware meets their physical constraints and performance requirements.

\subsection{\label{sec:method:evaluation-of-alternative-designs}Evaluation of Alternative Designs}

\subsection{\label{sec:method:proposed-verification-procedure}Proposed Verification Procedure}

\subsubsection{\label{sec:method:proposed-verification-procedure:software}Software}
The correct functioning of software is checked at every stage of development, using three levels, two of which are automated.

Software code not depending on hardware interaction is examined through unit tests, which verify that functions produce the correct output under all anticipated conditions. These tests are written before or simultaneously with the code to be tested.

The emergent, holistic functionality of the software (including user interface elements are outputs) can be tested using functional tests that provide simulated realistic input, possibly simulating hardware components, and examine the output.

The final test level is user tests, which include testing on the actual vessel in real and simulated environments, as well as any other manual testing.

These test levels are not performed in sequence; all testing levels should interact with the development process, and be re-evaluated whenever a notable change is made.

\begin{figure}
\includegraphics[width=100mm,natwidth=494,natheight=299]{"./image/software-testing"}
\caption[Software development cycle]{\label{fig:software-testing}Development cycle. Tests are written before or simultaneously with production code, and both automated and manual testing is run intermittently during development.}
\end{figure}

\subsubsection{\label{sec:method:proposed-verification-procedure:hardware}Hardware}
