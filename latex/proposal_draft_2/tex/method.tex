\subsection{\label{sec:method:theory}Theory}
There are numerous potential methods of detecting floating and undersea obstacles. Though these have rarely been applied to autonomous detection of objects, the functionality and expected usability of each system is discussed below. Also, general detection principles applicable to many detection schemes are discussed.


\subsection{\label{sec:method:proposed-design}Proposed Design}
The design that will be most likely to see all obstacles at approriate distances will rely on AIS for large boats, IR for small boats, radio reciever for nets and SAR for passive small obstacle detection. Two of these, AIS and SAR, are easily implemented solutions that rely solely on external data (no on-board sensors) and coding for positioning. The radio reciever simply needs to indentify the direction of a radio source to find a net marker, and avoiding that area should keep the boat away from any nets. The IR detection will be the most trickyof the individual systems to test and implement, and the true challenge will be integrating all four systems to appropriately manage data from each source.


\begin{itemize}
\item AIS, an internationally managed system for large boats to communicate positions, headings and speeds
\item IR beacon and detector, providing a 2D image for small boat detection
\item Radio reciever, to find radio transponders placed on net markers
\item SAR, realtime satellite-based radio imaging to find convergent currents and eddies where floating debris is likely to accumulate
\end{itemize}



\subsection{\label{sec:method:proposed-analysis}Proposed Analysis}
It is important to verify the correct functioning both of the hardware sensors and computation platform, and of the software analysis system. Beyond functionality, further examination will determine the limits of the capabilities of the device.
\subsubsection{\label{sec:method:proposed-analysis:software}Software}
Software can be examined at a basic level by following the principles of test driven development, and at a higher level by providing simulated or offline signals to the program.

\subsubsection{\label{sec:method:proposed-analysis:hardware}Hardware}
After meeting with our sponsor, we will develop a test plan to verify that our hardware meets their physical constraints and performance requirements.

\subsection{\label{sec:method:evaluation-of-alternative-designs}Evaluation of Alternative Designs}

\subsection{\label{sec:method:proposed-verification-procedure}Proposed Verification Procedure}