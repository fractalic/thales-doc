\subsection{\label{sec:method:theory}Theory}
There are numerous potential methods of detecting floating and undersea obstacles. Though these have rarely been applied to autonomous detection of objects, the functionality and expected usability of each system is discussed below.

\subsubsection{\label{sec:theory:radar}}



\subsection{\label{sec:method:proposed-design}Proposed Design}
It seems likely that a radar/lidar combination will be our primary detection method.  This will be reassessed after meeting with our sponsor.


\begin{itemize}
\item LIDAR system, where laser rangefinders are used to create a 3d map of the surrounding space
\item IR beacon and detector, providing a 2D image
\item Ultrasound. 3D image
\item Other things?
\end{itemize}



\subsection{\label{sec:method:proposed-analysis}Proposed Analysis}
It is important to verify the correct functioning both of the hardware sensors and computation platform, and of the software analysis system. Beyond functionality, further examination will determine the limits of the capabilities of the device.
\subsubsection{\label{sec:method:proposed-analysis:software}Software}
Software can be examined at a basic level by following the principles of test driven development, and at a higher level by providing simulated or offline signals to the program.

\subsubsection{\label{sec:method:proposed-analysis:hardware}Hardware}
After meeting with our sponsor, we will develop a test plan to verify that our hardware meets their physical constraints and performance requirements.

\subsection{\label{sec:method:evaluation-of-alternative-designs}Evaluation of Alternative Designs}

\subsection{\label{sec:method:proposed-verification-procedure}Proposed Verification Procedure}