Project Objectives are the intended goals of the project, and should not change as a result of circumstances that happen during the project.   Project Objectives can be used to drive the methods and deliverables for the project, which are the specific ways the group intends to achieve the Project Objectives for the sponsor.

Project Objectives ideally take the form of quantitative statements of project goals.  As much as possible, include quantitative information and target values for the desired performance of the system – this makes it possible to objectively examine each part of proposed solutions, and for all parties to understand the understand the true goals of the project.

As the project progresses, individual tasks may have to be changed given any number of unexpected events. Individual tasks can be changed, deliverables can change their form, but the Project Objectives as defined in this section are targets and goals which should remain fixed throughout the project.
