Our project is to develop an obstacle detection system for use by the Sailbot autonomous sailboat in its attempt to autonomously transit the Atlantic ocean, from Newfoundland to Ireland. Our system is intended to identify hazards, providing the position of the hazards to the Sailbot's route-planning system. Our current system, still under development, uses an infrared camera (FLIR Lepton) to identify floating obstacles.

Our team designed and assembled two waterproof and durable test rigs to evaluate the Lepton camera through on-water testing. As well as the camera, these test rigs housed battery packs, a microcomputer (Raspberry Pi) and an inertial motion unit (IMU; ST Discovery Board) to track the orientation of the camera. These test rigs are controlled by a Python script, developed in conjunction with the Sailbot team, which captures images, tags orientation data, and allows wireless access.

We initially planned to detect large obstacles such as container ships and fishing vessels at a range of 100m using an Automatic Identification System (AIS) receiver, but this task has been completed by other members of the Sailbot team. We focused instead on evaluating and optimizing the Lepton system. Through our tests, it became apparent that objects such as buoys and boats are detectable at close distances, less than 100m, and that salt buildup on the lens may pose a problem. As a result, we propose designs for the final camera enclosure to minimize water and salt accumulation on the lens.

The Sailbot team has met with some success in identifying floating obstacles captured on our test footage, however the low resolution of the camera makes many obstacles difficult to detect at great distance. We recommend the use of a higher-resolution camera, which has recently come within the Sailbot budget due to increased sponsorship. We furthermore expound the advantage of zinc selenide over plastic as an infrared-transparent lens. The cost of our design, with the current low-resolution camera, is less than \$1000.
