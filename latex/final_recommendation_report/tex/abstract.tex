Our project is to develop an obstacle detection system for use by the Sailbot autonomous sailboat in its attempt to autonomously transit the Atlantic ocean, from Newfoundland to Ireland. Our system is intended to identify hazards, providing the position of the hazards to the Sailbot's route-planning system. We plan to detect large obstacles such as container ships and fishing vessels at a range of 100m using an Automatic Identification System (AIS) receiver, which is required for all large vessels at sea. Smaller objects, such as boats below the minimum AIS size requirement, and rarely-occuring logs, and icebergs will be detected at a range of several meters but not exceeding 100m by an infrared camera. Nets will be detected using a radio reciever, as they are marked with transponders. This will rely on open source object identification algorithms, as well as our own code.

Our solution should have a low weight, not exceeding a few kilograms, and low power draw that does not exceed 2W on a regular basis and 36W at maximum. The power requirements exclude many commercial infrared cameras, though it is possible to selectively use a device or build a suitable device for less than a few hundred dollars. The solution will be tested using sample data gathered in the field; the Sailbot team currently has access to an IR camera and an AIS receiver which can be readily used to make test scenarios. The implementation of our system can be tested on the small scale Sailbot, for which it will be much easier to simulate the rough ocean conditions that the Atlantic-faring large Sailbot will encounter.

This project is split into three main milestones: being able to detect boats in good conditions, being able to detect small obstacles in good conditions, and being able to detect both sets of obstacles in rough conditions. The first two are relatively independent problems which will take the better part of two months to solve, while merging and upgrading these systems is allowed another month. This leaves one month at the end of the term to install the system on the Sailbot proper, finish any testing or optimization, and allow for unexpected challenges to be overcome. With collaboration from the Sailbot subteams, this project will be finished in ample time to ensure the Sailbot has the greatest chance of success in its August 2015 launch.