% This is a subsection: use only subsubsections in this file.
The objective of this project is to create a system capable of analyzing the dynamic nautical environment around a 5.5 meter sailboat. Hazards of any shape, posing a measureable threat to the boat's integrity, must be identifiable by the module at a distance appropriate for avoiding the hazard. The module must provide sufficient time for the sailboat to steer away from any object that might damage the boat; this time varies between obstacles as follows.

Larger objects, such as boats, buoys and icebergs, are the primary challenges for the obstacle detection system. The most massive (and dangerous) of these objects should be detected at a range greater than 100m; this range is reserved for large boats, tankers and icebergs. Small boats- in the size of private oceangoing fishing boats- and buoys should be detected at a minimum range of 50m. These ranges have been determined by examining the relative sizes and mobility of these objects and comparing that to the distance that the sailbot (at a maximum operating speed of 12 knots) would need to be sure to avoid the hazard.

Furthermore, due to power constraints, the module cannot use more than 36 W of power at any given time, and should not use more than 2 W sustained. Structural considerations prohibit a large mass from being placed high on the mast due to the counterweight moment needed to keep the boat upright, so the any parts of the detection system not located in the hull of the boat should be limited to a 0.5kg weight, while also minimzing aerodynamic losses. We have approximately two 1 ft cubed spaces in the hull of the boat to fit our equipment. 

Finally, the module must interface with the Raspberry Pi board that controls the routemaking of the autonomous sailboat to provide the relative position of detected obstacles; the avoidance and pathing system is under independent development by a Sailbot subteam.

In order to satisfy the weight and power requirements, infrared imaging was chosen as the optimal obstacle detection technique. Traditional marine sensing systems, such as radar and sonar, far exceeded all our contraints for even the smallest and lowest-power commercial systems. A long-wave infrared imaging FLIR Lepton camera was purchased, with a 50 degree field of view at a 80x60 resolution; over the course of this term, this Lepton was the focus of our efforts as we manipulated it to satisfy the above detection requirements.

%Several potential types of obstacles are left out of the project objectives because they are exceedingly rare and have been deemed to be outside of the scope of this project. Among these obstacles are small floating debris, such as logs, seaweed clumps, floating sea birds, and abandoned fishing equipment - we will not attempt to detect such objects.