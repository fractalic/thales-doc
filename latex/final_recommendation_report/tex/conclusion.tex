The development of an automated machine vision-based obstacle detection system for the Sailbot's 2015 MicroTransatlantic Challenge is a difficult problem. Our project has helped the Sailbot team evaluate the effectiveness of infrared-based obstacle detection, initially selected due to its low cost and ease of implementation, despite the dearth of literature describing its use in similar challenges.

Throughout the term, we developed Test Rigs, to protect the FLIR Lepton infrared camera during testing, and providing the requisite power as well as hardware and software control in a convenient, waterproof package. These Test Rigs will also inform the final design of the distributed housing, where the infrared camera will be mounted to a tripod on the boat and the electronics will be mounted in the hull of the boat, several meters away.

Our testing showed the effectiveness of the tripod mount for the camera, confirmed the functionality of serial communication between the Raspberry Pi and Lepton over several meters of cable, and our design and development led to a sturdy baseplate to which a small infrared camera can be secured. This baseplate can then be secured to any other compatible module, such as a Test Rig.

We also provided several battery packs, an inertial motion unit, and plan to supply a zinc selenide infrared-transparent lens to the Sailbot team, to assist in testing and to be incorporated into the final design. In particular, we are currently assisting the Sailbot team in decoding data from the inertial motion unit, and design a lens cleaning system to keep clear the camera's field of view.

In cooperation with the Sailbot team, we were able to identify some floating objects using machine vision algorithms on the low-resolution Lepton camera images acquired with our Test Rig. Finally, we recommend the use of the higher resolution FLIR Quark camera, which will provide greater detection range of floating objects.
