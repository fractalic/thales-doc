% This is a subsection: use only subsubsections in this file.
\iffalse
Experimental Equipment / Flow Diagram / Algorithms
This section should include figures and photos of the experimental apparatus and team-fabricated elements, but should not show detailed drawings of the components. Technical details and drawings can be included in an appendix.
\fi

\subsubsection{\label{sec:discussion:equipment:hardware}Test Hardware}

\begin{figure}
\centering
\begin{subfigure}{.65\textwidth}
  \centering
  \includegraphics[width=.95\linewidth]{"./image/20150219-_BEN2653"}
  \caption{Close up of Test Rig 1.0}
  \label{fig:test_rig_1.0_tofino:sub1}
\end{subfigure}%
\begin{subfigure}{.35\textwidth}
  \centering
  \includegraphics[width=.95\linewidth]{"./image/20150219-_BEN2655"}
  \caption{Test Rig 1.0, clamped to a tripod secured to the bow of a small boat.}
  \label{fig:test_rig_1.0_tofino:sub2}
\end{subfigure}
\caption[Test Rig 1.0 open-ocean test.]{Two views of an open-ocean test at Tofino of the FLIR Lepton, contained in Test Rig 1.0.}
\label{fig:test_rig_1.0_tofino}
\end{figure}

In order to evaluate the FLIR Lepton infrared camera, it was important to secure it in a waterproof housing. This housing, dubbed the Test Rig 1.0, was constructed from a large oval plastic container with a waterproof gasket-sealing lid. A single hole was drilled in the side to accommodate a button to start and stop the recording of video from the Lepton.

\begin{figure}
\centering

\begin{subfigure}{.50\textwidth}
\centering
\includegraphics[width=.90\linewidth]{"./image/lepton"}
\caption[FLIR Lepton]{\label{fig:lepton}The FLIR Lepton infrared camera (black) mounted in a green breakout board developed by Pure Engineering \cite{pure-eng}. (Image CC BY-NC-SA 3.0 by SparkFun.)}
\end{subfigure}%
\begin{subfigure}{.50\textwidth}
\centering
\includegraphics[width=.90\linewidth]{"./image/raspberry-pi"}
\caption[Raspberry Pi]{\label{fig:rpi}The Raspberry Pi Model B+, a microcomputer used by the Sailbot team for all control functions, as well as by our group to run the Test Rig \cite{rpi}. (Image CC BY-NC-SA 3.0 by SparkFun.)}
\end{subfigure}

\begin{subfigure}{.45\textwidth}
\centering
\includegraphics[width=.90\linewidth]{"./image/discovery-board"}
\caption[Discovery Board]{\label{fig:discovery-board}The ST Discovery Board, with integrated IMU unit. (Image from ST.)}
\end{subfigure}
\caption[Equipment used in the Test Rig.]{\label{fig:rpi-lepton}The Raspberry Pi and FLIR Lepton used in the Test Rig.}
\end{figure}

Figure \ref{fig:test_rig_1.0_tofino} shows Test Rig 1.0 being tested at Tofino in February. (The Test Rig interior has since been refined and restructured in version 2.0). This test included data collection in sunshine and rain, and in broad daylight, evening twilight, and early night.

As well as providing water protection, the plastic container was large enough to hold a battery pack, Raspberry Pi computer, and  the Lepton camera. We cut a window in the container and covered it with infrared-transparent film to provide a viewport for the camera - this will be replaced with a more robust zinc selenide window in the complete obstacle detection system.  Version 2.0 of the Test Rig features a laser-cut plastic interior support structure to organize all cables and devices, and accommodate a large inertial measurement unit (IMU).  

\subsubsection{\label{sec:discussion:equipment:software}Test Software}
The Test Rig is controlled by a straightforward Python script, shown in Figure \ref{fig:test-rig-software}. The script responds to button presses, starting and stopping the recording of video from the Lepton. 60 by 80 pixel frames are grabbed from the Lepton approximately 20 times per second at 16 bit depth. These high bit depth images, which high tonal gradation resolution, are then converted to an 8-bit video stream. Version 2.0 of the Test Rig hardware includes an IMU, and this data is polled and logged for each frame captured.

As information is lost in the conversion of 16-bit source images to 8-bit video, the original 8-bit images are stored alongside the video output. As well, the bit-mapping for the video is changed dynamically - the darkest and brightest regions in the video are assigned minimal and maximal values respectively in the 8-bit encoding, whether or not these regions do not take on maximal values in the 16-bit images. In effect, therefore, the video exposure is computed frame-by-frame. This maximizes the contrast ratio in the video, though this approach does not always produce the best frames for analysis.

\begin{figure}
\centering
\includegraphics[width=\linewidth]{"./image/button_py"}
\caption[Test Rig software flow]{\label{fig:test-rig-software}The flow of the Test Rig control software.}
\end{figure}

\subsubsection{\label{sec:discussion:equipment:architecture}Software Architecture and Routemaking Integration}

\paragraph{\label{sec:discussion:equipment:architecture:obstacledetection}Obstacle Detection}
Our software starts by identifying obstacles in a single frame of video.  First, it applies Erosion and Dilation filters to reduce noise, then uses OpenCV's Canny Edge Detection to find contours.  Finally, it draws bounding boxes around areas with a high density of contours - these are considered to be obstacles.

Work on this part of the system is currently on hold, as UBC Sailbot has been offered a donation of a higher-resolution infrared camera, the FLIR Quark.  This will significantly improve our detection capabilities, but is likely to require different filters and detection techniques from the Lepton.

\paragraph{\label{sec:discussion:equipment:architecture:obstaclelocation}Obstacle Location}After identifying an obstacle in an image, the next step is to determine its location in space to allow Sailbot to avoid it.  We will not attempt to estimate the distance to an obstacle, only the direction.  Figure \ref{fig:horizon-finding} shows our procedure for determining the direction of an obstacle - each obstacle's position in the image is projected onto the horizon line, giving its angle relative to Sailbot's path.

This approach depends on reliable identification of the horizon line.  Waves, rain, and fog make finding the horizon from the image unreliable at best.  Instead, we will use the IMU to determine Sailbot's orientation, allowing us to infer the position of the horizon - the Test Rig 2.0 includes an IMU to support this.  We will combine angular rate integration of the gyroscope data (to respond rapidly to changes in orientation) with a moving average of the direction of gravity as measured by the accelerometer (to provide a reliable reference point to correct for drift in the gyro measurements) to determine orientation.

\begin{figure}
\includegraphics[width=100mm,natwidth=1203,natheight=627]{"./image/horizon_finding"}
\caption[UBC Sailbot.]{\label{fig:horizon-finding}Direction of obstacles is determined by projecting them onto the horizon line}
\end{figure}

\paragraph{\label{sec:discussion:equipment:architecture:routemaking}Routemaking Integration}
After locating obstacles in space, Sailbot needs to plot a course around them.  The UBC Sailbot team is currently developing the routemaking software that will control Sailbot's navigation.  This software assigns "hazard values" to a grid of points and plots a course that minimizes the encountered hazards, allowing the Sailbot to avoid known, fixed obstacles such as islands.  Our obstacle detection system will be integrated into this hazard map, allowing the same routemaking software to avoid detected obstacles.

When an obstacle is identified in a frame, we will add to the hazard value of points in a cone centered on its estimated angle, as shown in Figure \ref{fig:routemaking}.  These hazard values will decrease over time - false detections will fade out, but real obstacles will be visible over several frames, increasing their hazard value enough to cause routemaking to steer around them.

\begin{figure}
\includegraphics[width=160mm,natwidth=1203,natheight=627]{"./image/routemaking"}
\caption[UBC Sailbot.]{\label{fig:routemaking}Routemaking}
\end{figure}