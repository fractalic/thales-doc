\begin{enumerate}
  \item Switch from the FLIR Lepton to the Quark2 camera.  The Quark2's resolution of 640x480 pixels is a major improvement over the Lepton's 80x60, and will substantially increase the range and reliability of obstacle identification. The housing for the infrared camera was designed with the intent that it's easy to modify to accomodate for a Quark camera, and the software flow to take images into obstacle detection has been built. New interfacing protocol will be needed, as the Quark camera has seperate drivers and communication protocol from that used for the Lepton, but FLIR provides software to handle this interface. Upgrading to this camera will allow for obstacle detection from much greater ranges, be less prone to failure (as the Quark has a shutter, and so better flat-field correction functions to account for pixel drift) and, as it has greater resolution, will be able to work better in foggy or rainy conditions as it can gather more information from a given field of view.
  \item Refine obstacle identification software based on video from new camera, to make use of improved resolution. The current iteration of the obstacle detection software uses minimal smoothing/ noise reduction techniques, as the information from the Lepton is already in a highly lossy state. The Quark will require more robust smoothing functionality to accommodate for the increased information density, particularly in the water. Higher resolution pictures of the water show more of the high-intensity sun reflections that the images taken with the Lepton (see Figure \ref{fig:defects:sub3} for more information).
\end{enumerate}
