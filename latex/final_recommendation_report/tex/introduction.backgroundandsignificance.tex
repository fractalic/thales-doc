% This is a subsection: use only subsubsections in this file.
This project was sponsored by the UBC Sailbot team, with the goal of overcoming a significant challenge to autonomous robotic vessels on open water: collision avoidance. Previous attempts made to cross the Atlantic ocean with an autonomous boat have often been sunk by collisions with ships or floating debris. Our task has been to develop a system capable of detecting obstacles and passing their location to the Sailbot control systems for route determination.

The boat is a 4m scale-up of the 1.5m boat successfully used by the Sailbot team in shorter autonomous navigation challenges. Figure \ref{fig:sailbot} shows this smaller boat.

\begin{figure}
\includegraphics[width=120mm,natwidth=1203,natheight=627]{"./image/sailbot"}
\caption[UBC Sailbot.]{\label{fig:sailbot}UBC Sailbot, used in several Sailbot competitions. \cite{ubc-sailbot__image}}
\end{figure}

The Microtransat Challenge \cite{transat-history} is a project that has been running for several years where participants are required to build an autonomous sailboat capable of sailing from North America to Europe, or in the opposite direction. Previous attempts at the Microtransat challenge have failed mostly through processes other than collision, but the few that have sunk due to collisions have primarily been caught in fishing nets \cite{transat-history}. Anecdotal evidence gathered from fishermen and a sailor who crossed the Atlantic seven times in a schooner tells us that the vast majority of all obstacles the Sailbot may encounter at sea will be found within a few days' sailing of shore; this is indicated by the Microtransat history, where those boats that failed due to collision or unknown circumstances did so long before crossing into international waters. This is also confirmed by Figure \ref{fig:ais-snapshot}, which shows the typical density of AIS-equipped vessels, highest near the shore, as of April 2015\cite{marine-traffic}.

\begin{figure}[H]
\centering
\includegraphics[width=150mm,natwidth=792,natheight=313]{"./image/AIS_density_north_atlantic_3"}
\caption[Instantaneous position of AIS vessels.]{\label{fig:ais-snapshot}Density map of AIS-equipped vessels.  The density is low away from shore and outside major shipping routes.}
\end{figure}

The UBC Sailbot team will make their attempt in late 2015, following the west to east route, which begins off the coast of Newfoundland, and ends near the Irish coast, as shown in figure \ref{fig:w-e_start-finish}. The boat must sail on its own for 74 kilometers (40 nautical miles) before crossing the start line, eliminating the possibility of towing the boat through the most debris-filled region close to shore.

\begin{figure}[H]
\centering
\includegraphics[width=150mm,natwidth=667,natheight=264]{"./image/start-finish_map"}
\caption[Microtransat keypoints.]{\label{fig:w-e_start-finish}Start and finish lines for the West to East route of the Microtransat \cite{transat__w-e_start-finish}. Start line is shown in blue. The boat must sail on its own toward the start line. }
\end{figure}
