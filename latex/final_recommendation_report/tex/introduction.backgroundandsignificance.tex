% This is a subsection: use only subsubsections in this file.
This project was sponsored by the UBC Sailbot team, with the goal of overcoming a significant challenge to autonomous robotic vessels in open water: collision avoidance. UBC Sailbot is competing in the Microtransat Challenge \cite{transat-history}, which requires participants to build an autonomous sailboat capable of crossing the Atlantic Ocean.  There have been nine attempts made at this challenge since its start in 2010, but none have been successful - several of these failed attempts were due to collisions \cite{transat-history}.  Our task has been to develop a system capable of detecting obstacles and passing their location to the Sailbot control systems for route determination.

The UBC Sailbot team will make their attempt in late 2015, following the west to east route, which begins off the coast of Newfoundland and ends near the Irish coast, as shown in Figure \ref{fig:route_density}. The boat must sail on its own for 74 kilometers (40 nautical miles) before crossing the start line, eliminating the possibility of towing the boat through the most debris-filled region close to shore.

The boat is a 5.5m scale-up of the 1.5m boat successfully used by the Sailbot team in shorter autonomous navigation challenges. Figure \ref{fig:sailbot} shows this smaller boat.  Ocean vessels typically use radar for obstacle detection, but the power and weight requirements of a radar system are prohibitive for such a small boat - our system uses an infrared camera instead. We have found a few groups working on similar systems \cite{ODU-boat-IR-detection}, but there has been relatively little research on IR-based marine obstacle detection.

\begin{figure}
\includegraphics[width=120mm,natwidth=1203,natheight=627]{"./image/sailbot"}
\caption[UBC Sailbot.]{\label{fig:sailbot}UBC Sailbot's previous autonomous sailboat \cite{ubc-sailbot__image}}
\end{figure}

\begin{figure}
\centering
\includegraphics[width=150mm,natwidth=792,natheight=313]{"./image/AIS_density_sailbot_route"}
\caption[Microtransat route and vessel density]{\label{fig:route_density}Start and finish lines for the West to East route of the Microtransat challenge \cite{transat__w-e_start-finish}, with density map of AIS-equipped vessels.  The density is low away from shore and outside major shipping lanes, but vessels are present throughout the expected route.}
\end{figure}

