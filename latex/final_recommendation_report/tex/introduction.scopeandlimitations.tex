% This is a subsection: use only subsubsections in this file.

\iffalse
From the guideline:
This is where the subject of the report can be specified fully; indicate not only what you are examining, but also when and where, as appropriate. As well as knowing what to expect in your report, your reader needs to know what not to expect. Indicate the kinds of problems, places, times, and personnel that are not considered and the impact these omissions and constraints may have on your results. You may also want to explain why these limitations have been necessary.

Notes:
what does the report consider?
-test rig, design and development
-testing IR camera feasibility on two days
-developing a control system for the test rig, allowing for video capture
-the team might be able to get a better camera
-long cables seem to be fine

what didn't we do?
-we don't evaluate which algorithms are best
--we spent a lot of time getting ready to gather data (getting test rigs built and working)
--lots of time on video processing systems to get good contrast
-the lepton cannot see small objects such as logs
-we don't provide a final housing
--sailbot might use our lepton holder, but will need to design a smaller waterproof case for it

impact of omissions:
-we can suggest a lepton holder, but not a final housing
-our work will help the sailbot team test algorithms and gather data, but we can't comment on the best algorithms
\fi

This report examines the Test Rig -- a housing designed to keep the an infrared camera safe during testing and data collection -- including the hardware and software design work required by the Test Rig. The feasibility of a low resolution infrared camera (FLIR Lepton) for the detection of floating obstacles at sea is assessed, relying on data gathered from multiple days on open and bay waters. As well, the usefulness of higher resolution infrared cameras (such as the FLIR Quark) is analysed in accompaniment with practical considerations such as an appropriate mounting point for the obstacle detection system. Finally, the obstacle detection system is designed to detect obstacles in front the boat, so the detection of obstacles approaching from behind or the side is not considered.

However, the feasibility of various computer vision algorithms is not assessed, as the project timeline was lengthened to allow time to construct equipment for data collection and to collect the data under realistic open-ocean conditions. Also, the mount point and camera housing for implementation on the final sailbot is considered, but the precise design has not been finalized. Finally, we do not compare the relative detectability of certain objects, as they are exceedingly rare and unlikely to be encountered. Among these obstacles are small floating debris such as logs, seaweed clumps, floating sea birds, and abandoned fishing equipment.
