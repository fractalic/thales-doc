While sailors have used navigation aides for many years, only recently has there been an interest in developing autonomous vehicles that could navigate themselves around the ocean for extended periods of time. While there is a wealth of information available on object detection for autonomous surface vehicles \cite{detection-offroad} \cite{optical-flow-detection} \cite{unmanned-ground-vehicles}, many of these techniques involve laser rangefinding or stereo vision in static, sterile environments. None of these techniques are easily applicable to an ocean environment, where the platform and environment are never static, waves and fog mean a constant air/water boundary has to be considered, salt posses a corrosive threat and marine life may grow on or distort the imaging medium. 

Autonomous underwater vehicles use forward looking sonar but the moving air/water medium for a boat on the surface makes sonar difficult to use when not completely submersed. While profiling sonar has been used in a research capacity on a surface boat\cite{ASV-sonar} the profiling system was only effective when tested on a small lake with a slow-moving boat, for objects at a distance of <30m. This case is a far cry from the conditions that the Sailbot will experience, and furthermore cost considerations rule out sonar obstacle detection systems for us.

The most applicable research we found to our work with the Lepton infrared camera was done at the Old Dominion University in Viriginia, by their Computational Intelligence \& Machine Vision Labratory. The ODU lab had developed an algorithm to count boats passing into a harbour by use of an infrared camera, primarily through the use of adaptive thresholding and training a machine learning algorithm on a large database of boat images; however, their work could be fooled on even slightly wavey water, as it was designed with harbour conditions in mind. Watching the sample video on their site shows the algorithm misidentifying water as an obstacle when there are even minor waves.\cite{ODU-boat-IR-detection}

