\textbf{Install and understand the basic principles of OpenCV.} This is the first phase of learning the software that we will use to analyse our infrared image data for floating obstacles. To complete this milestone, we plan to set up OpenCV on our computers, and read through the first three chapters of Learning OpenCV \cite{kaehler__learning-opencv}.\\\textit{Completion date: January 14, 2015.}
\par
\textbf{Learn to analyse images in OpenCV.} Over several weeks, we plan to deepen our knowledge of OpenCV by working closely with Sailbot to work through Learning OpenCV \cite{kaehler__learning-opencv}. Our combined team of approximately eight people will divide the book into sections, where each member will learn one section and provide a tutorial to the rest of the group.\\\textit{Completion date: January 28, 2015.}
\par
\textbf{Build a testing rig for an infrared camera.} In order to gather situational data from an infrared camera, the Sailbot team requires a rig that can be handheld or mounted to a floating mockup of the boat. The rig should support an infrared camera (precise model to be determined), as well as a power supply, and processor and storage system such as a Raspberry Pi.\\\textit{Completion date: February 6, 2015.}
\par
\textbf{Finish the AIS receiver.} The AIS receiver is a relatively straightforward device to verify. As described in section \ref{sec:method:proposed-design:ais}, the AIS device we will use comes with software to analyse its output. Our work is to ensure that the GPSD library produces the same results, by testing the receiver in the vicinity of ships in the Vancouver hardware during one or two afternoons. Once verified, we provide our driver code (consisting of the GPSD library and any other functions required to interact with the AIS receiver) to the Sailbot team for inclusion in their software.\\\textit{Completion date: February 6, 2015.}
\par
Detect boats in calm conditions
Detect small obstacles in good conditions
Detect boats and small obstacles in bad conditions
